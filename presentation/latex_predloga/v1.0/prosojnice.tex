\documentclass[xcolor=dvipsnames,compress]{beamer}
%\usepackage[latin1]{inputenc}
\usepackage{subfigure}
\usepackage{mpmulti}
\usepackage{epstopdf}

%
% Osnovna tema prosojnic je ULFRI
% Ce zelite uporabiti mini-frames temo (kot jo imata Berlin ali Warsaw) in jo radi uporabljajjo matematiki, 
% izberite \usetheme[compress]{ULFRIframes}
%
\usetheme[height=9mm]{ULFRI}
%\usetheme[compress]{ULFRIframes}

\title[FbHash]
{FbHash: shema za izra\v{c}un podobnosti datotek v digitalni forenziki}


\author[Knez, Mežnar, Pegan] % (optional, use only with lots of authors)
{Timotej Knez\inst{1},  Sebastian Mežnar\inst{2}, Jasmina Pegan\inst{1}}
% - Give the names in the same order as the appear in the paper.
% - Use the \inst{?} command only if the authors have different
%   affiliation.

\institute[UL FRI, UL FMF] % (optional, but mostly needed)
{
  \inst{1}%
  Fakulteta za računalništvo in informatiko\\
  Univerza v Ljubljana
  \and
  \inst{2}%
  Fakulteta za matematiko in fiziko\\
  Univerza v Ljubljana
}
% - Use the \inst command only if there are several affiliations.
% - Keep it simple, no one is interested in your street address.

\date[DigFor20] % (optional, should be abbreviation of conference name)
{}
% - Either use conference name or its abbreviation.
% - Not really informative to the audience, more for people (including
%   yourself) who are reading the slides online


%
% Ce uporabite mini-frames temo, potem vkljucite logo na naslovnici s spodnjim ukazom:
%
% logo of my university
%\titlegraphic{\includegraphics[height=20mm]{znakULFRIbeamer.png}}


% If you wish to uncover everything in a step-wise fashion, uncomment
% the following command: 

%\beamerdefaultoverlayspecification{<+->}


\begin{document}

\begin{frame}
\titlepage
\end{frame}
%%%%%%%%%%%%%%%%%%%%%%%%%%%%%%%%%%%%%%%%%%%%%%%%%%%%%%%%%%%%%%%%%%%%%%%%%%%%%%%%
\section{Kazalo}
\label{sec:kazalo}
%%%%%%%%%%%%%%%%%%%%%%%%%%%%%%%%%%%%%%%%%%%%%%%%%%%%%%%%%%%%%%%%%%%%%%%%%%%%%%%%

\begin{frame}{Kazalo}
\begin{itemize}
  \item Uvod
  \item Sorodna dela
  \item Algoritem
  \item Eksperimenti v članku
  \item Naši eksperimenti
  \item Zaključek
\end{itemize}
\end{frame}


%%%%%%%%%%%%%%%%%%%%%%%%%%%%%%%%%%%%%%%%%%%%%%%%%%%%%%%%%%%%%%%%%%%%%%%%%%%%%%%%
\section{Uvod}
\label{sec:uvod}
%%%%%%%%%%%%%%%%%%%%%%%%%%%%%%%%%%%%%%%%%%%%%%%%%%%%%%%%%%%%%%%%%%%%%%%%%%%%%%%%

\begin{frame}{Uvod}
\begin{itemize}
	\item Avtomatizacija preiskave datotek
	\item Algoritmi za iskanje približnega ujemanja
	\begin{itemize}
		\item ssdeep, sdhash, FbHash
	\end{itemize}
	\item Prispevki članka
	\begin{itemize}
		\item odporna shema, dve različici, analiza varnosti
	\end{itemize}
	\item Implementacija algoritma
	\begin{itemize}
		\item različne funkcije za uteževanje
	\end{itemize}
	\item Testiranje na istih množicah kot v članku
\end{itemize}
\end{frame}



%%%%%%%%%%%%%%%%%%%%%%%%%%%%%%%%%%%%%%%%%%%%%%%%%%%%%%%%%%%%%%%%%%%%%%%%%%%%%%%%
\section{Sorodna dela}
\label{sec:sorodna}
%%%%%%%%%%%%%%%%%%%%%%%%%%%%%%%%%%%%%%%%%%%%%%%%%%%%%%%%%%%%%%%%%%%%%%%%%%%%%%%%

\begin{frame}{Sorodna dela}
\begin{itemize}
	\item ssdeep
	\begin{itemize}
		\item Temelji na algoritmu spamsum
		\item Funkcija z drsečim oknom
	\end{itemize}
	\item sdhash
	\begin{itemize}
		\item Statistično najmanj verjetni deli datoteke
		\item Bloomovi filtri
	\end{itemize}
	\item MRSH-v2
	\begin{itemize}
		\item Multi-resolution similarity hashing
		\item Funkcija z drsečim oknom, Bloomovi filtri
	\end{itemize}
	\item mvHash-B
	\begin{itemize}
		\item Krajšanje zapisa: glasovanje, kompakten zapis
		\item Bloomovi filtri
	\end{itemize}
\end{itemize}
\end{frame}
%

%%%%%%%%%%%%%%%%%%%%%%%%%%%%%%%%%%%%%%%%%%%%%%%%%%%%%%%%%%%%%%%%%%%%%%%%%%%%%%%%
\section{Algoritem}
%%%%%%%%%%%%%%%%%%%%%%%%%%%%%%%%%%%%%%%%%%%%%%%%%%%%%%%%%%%%%%%%%%%%%%%%%%%%%%%%

\begin{frame}{Algoritem}
\begin{itemize}
    \item FbHash-B za nestisnjene in FbHash-S za stisnjene datoteke
    \item Glavna komponenta so deli, sestavljeni iz $k$ zlogov
\end{itemize}
 \begin{figure}[h]
     \includegraphics[width=0.7\columnwidth]{figs/chunks.png}
	\caption{Oblika delov datoteke}
     \end{figure}
\end{frame}

\begin{frame}{FbHash-B}
Algoritem FbHash-B: 
    \begin{itemize}
    \item[1] Za vsak del datoteke izračunaj zgoščeno vrednost in preštej pojavitve.
    \item[2] Uravnoteži dele datoteke glede število pojavitev
    \item[3] Preštej in uravnoteži pojavitve delov v bazi podatkov
    \item[4] Izračunaj končno oceno za datoteko iz uteži
    \end{itemize}
\end{frame}

\begin{frame}{Funkcije za izračun uteži}
\begin{itemize}
\item Funkcije za uteži delov znotraj datoteke:
    \begin{itemize}
    \item $ W_{ch_i}^{D}(ch f_{ch}^D) = 1 + \log_{10}\left(\frac{ch f_{ch}^D}{n}\right)$
    \item $W_{ch_i}^{D}(ch f_{ch}^D) = \log_2\left(1 + \frac{ch f_{ch_i}^D}{n}\right)$
    \item $W_{ch_i}^{D}(ch f_{ch}^D) = \frac{ch f_{ch_i}^D}{n}$
    \end{itemize}
\item Funkcije za uteži delov v podatkovni bazi
\begin{itemize}
    \item $docw_{ch_i}^{D}(df_{ch}) = \log_{10}\left(\frac{m}{df_{ch}}\right)$
    \item $docw_{ch_i}^{D}(df_{ch}) = 1 - \log_{10}\left(\frac{df_{ch}}{m}\right)$
    \end{itemize}
\end{itemize}
\end{frame}

\begin{frame}{Algoritem}
    \begin{itemize}
    \item Primerjava datotek s kosinusno razdaljo
    \end{itemize}
   \[Similarity(D_1, D_2)=\frac{\sum_{i=0}^{n-1} W_{ch_i}^{D_1} \cdot W_{ch_i}^{D_2}}{\sqrt{\sum_{i=0}^{n-1} (W_{ch_i}^{D_1})^2}\cdot\sqrt{\sum_{i=0}^{n-1} (W_{ch_i}^{D_2})^2}}\cdot 100\]
   \begin{itemize}
    \item FbHash-S uporabi FbHash-B na razširjenih datotekah
    \end{itemize}
\end{frame}

%%%%%%%%%%%%%%%%%%%%%%%%%%%%%%%%%%%%%%%%%%%%%%%%%%%%%%%%%%%%%%%%%%%%%%%%%%%%%%%%
\section{Eksperimenti v članku}
\label{sec:eksoni}
%%%%%%%%%%%%%%%%%%%%%%%%%%%%%%%%%%%%%%%%%%%%%%%%%%%%%%%%%%%%%%%%%%%%%%%%%%%%%%%%

\begin{frame}{Eksperimenti v članku}
    \begin{itemize}
    \item Algoritem varen proti napadu z aktivnim napadalcem
    \item Detekcija fragmentov
    \item Korelacija skupnega dela datoteke
    \end{itemize}
\end{frame}

\begin{frame}{Detekcija fragmentov}
   \begin{figure}[ht!]
        \label{fig:subfigures}
        \begin{center}
            \subfigure{%
                \label{fig:first}
                \includegraphics[width=0.5\textwidth]{figs/frag_det_text.png}
                %\caption{Poptavljene uteži}
            }%
            \subfigure{%
               %\label{fig:second}
                \includegraphics[width=0.5\textwidth]{figs/frag_det_docx.png}
                %\caption{}
            }
        \end{center}
       \caption{Rezultati detekcije fragmentov iz članka}
    \end{figure}
\end{frame}

\begin{frame}{Korelacija skupnega dela datoteke}
   \begin{figure}[ht!]
        \label{fig:subfigures}
        \begin{center}
            \subfigure{%
                \label{fig:first}
                \includegraphics[width=0.65\textwidth]{figs/cor_text.png}
                %\caption{Poptavljene uteži}
            }\\%
            \subfigure{%
               %\label{fig:second}
                \includegraphics[width=0.65\textwidth]{figs/cor_docx.png}
                %\caption{}
            }
        \end{center}
       \caption{Rezultati korelacije skupnega dela datoteke iz članka}
    \end{figure}
\end{frame}

%%%%%%%%%%%%%%%%%%%%%%%%%%%%%%%%%%%%%%%%%%%%%%%%%%%%%%%%%%%%%%%%%%%%%%%%%%%%%%%%
\section{Naši eksperimenti}
\label{sec:eksmi}
%%%%%%%%%%%%%%%%%%%%%%%%%%%%%%%%%%%%%%%%%%%%%%%%%%%%%%%%%%%%%%%%%%%%%%%%%%%%%%%%
\begin{frame}{Naši eksperimenti}
    \begin{itemize}
        \item Implementacija algoritma v jeziku python
        \item Implementacija FbHash-S in FbHash-B
        \item Testiranje na prosti zbirki dokumentov t5 (http://roussev.net/t5/t5.html)
        \item Testiranje delovanja algoritma pri različnih funkcijah za izračun uteži
    \end{itemize}
    \end{frame}
    
    \begin{frame}{Postopek testiranja}
    \begin{itemize}
        \item Generiranje datotek z znanim ujemanjem
        \begin{itemize}
            \item Vzamemo dve datoteki iz zbirke
            \item Naključni blok prve vstavimo v drugo
            \item Primerjamo prvo datoteko z novo nastalo datoteko
        \end{itemize}
        \item Algoritem naučimo na celotni zbirki dokumentov
        \item Z algoritmom primerjamo ustvarjena dokumenta
        \item Opazujemo
        \begin{itemize}
            \item Delež datotek z zaznanim ujemanjem
            \item Povprečno oceno ujemanja glede na resnično ujemanje
            \item F-oceno
        \end{itemize}
    \end{itemize}
\end{frame}

\begin{frame}{Rezultati testiranja}
    \begin{itemize}
        \item Izračun uteži kot v izvornem članku
        \item $ W_{ch_i}^{D}(ch f_{ch}^D) = 1 + \log_{10}\left(\frac{ch f_{ch}^D}{n}\right)$
        \item F ocena: 0.94
        \item Rezultati skladni s člankom
    \end{itemize}
    
    \begin{figure}[ht!]
        \label{fig:subfigures}
        \begin{center}
            \includegraphics[width=0.8\textwidth]{figs/njihov_majhen.png}
        \end{center}
    \end{figure}
    
\end{frame}

\begin{frame}{Rezultati testiranja}
    \begin{itemize}
        \item Uporaba različnih uteži
        \begin{itemize}
            \item Levo: $W_{ch_i}^{D}(ch f_{ch}^D) = \log_2\left(1 + \frac{ch f_{ch_i}^D}{n}\right)$
            \item Desno: $W_{ch_i}^{D}(ch f_{ch}^D) = \frac{ch f_{ch_i}^D}{n}$
        \end{itemize}
    \end{itemize}
    \begin{figure}[ht!]
        \label{fig:subfigures}
        \begin{center}
            \subfigure{%
                \label{fig:first}
                \includegraphics[width=0.5\textwidth]{figs/ch_tfidf_weight_majhna.png}
                %\caption{Poptavljene uteži}
            }%
            \subfigure{%
               %\label{fig:second}
                \includegraphics[width=0.5\textwidth]{figs/ch_freq_weight.png}
                %\caption{}
            }
        \end{center}
       \caption{Testiranje delovanja algoritma z različnimi utežmi}
    \end{figure}
    
\end{frame}

\begin{frame}{Rezultati testiranja}
    \begin{itemize}
        \item Razpoznavanje brez zbirke datotek
        \item F ocena: 0.87
        \item Dovolj dober rezultat, da bi lahko algotitem uporabili tudi brez zbirke datotek.
    \end{itemize}
    
    \begin{figure}[ht!]
        \label{fig:subfigures}
        \begin{center}
            \includegraphics[width=0.8\textwidth]{figs/brez_courpusa.png}
        \end{center}
    \end{figure}
    
\end{frame}




%%%%%%%%%%%%%%%%%%%%%%%%%%%%%%%%%%%%%%%%%%%%%%%%%%%%%%%%%%%%%%%%%%%%%%%%%%%%%%%%
\section{Zaključek}
\label{sec:zakljucek}
%%%%%%%%%%%%%%%%%%%%%%%%%%%%%%%%%%%%%%%%%%%%%%%%%%%%%%%%%%%%%%%%%%%%%%%%%%%%%%%%

\begin{frame}{Zaključek}
    \begin{itemize}
        \item Pregled predlaganega algoritma ter sorodnih del
        \item Implementacija opisanega algoritma
        \item Potrditev rezultatov iz članka
        \item Predlagali smo različne načine izračunov uteži
        \item Ugotovili smo, da je algoritem mogoče uporabiti tudi brez zbirke datotek
    \end{itemize}
\end{frame}



\end{document}